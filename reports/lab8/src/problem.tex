 
\CWHeader{Лабораторная работа \textnumero 8}

\CWProblem{
	Откорм бычков.
	
	Бычкам дают пищевые добавки, чтобы ускорить их рост. Каждая добавка содержит некоторые из $N$ действующих веществ. Соотношения количеств веществ в добавках могут отличаться.
	
	Воздействие добавки определяется как $c_1 a_1 + c_2 a_2 + \dots +c_N a_N$, где $a_i$ — количество i-го вещества в добавке, $c_i$ – неизвестный коэффициент, связанный с веществом и не зависящий от добавки. Чтобы найти неизвестные коэффициенты $c_i$, Биолог может измерить воздействие любой добавки, использовав один её мешок. Известна цена мешка каждой из $M (M \leq N)$ различных добавок. Нужно помочь Биологу подобрать самый дешевый наобор добавок, позволяющий найти коэффициенты $c_i$. Возможно, соотношения веществ в добавках таковы, что определить коэффициенты нельзя.
	
	
	{\bfseries Формат ввода:} В первой строке текста – целые числа $M$ и $N$ ; в каждой из следующих $M$ строк записаны $N$ чисел, задающих соотношение количеств веществ в ней, а за ними – цена мешка добавки. Порядок веществ во всех описаниях добавок один и тот же, все числа – неотрицательные целые не больше 50. 
	
	{\bfseries Формат вывода:} Вывести -1 если определить коэффциенты невозможно, иначе набор добавок (и их номеров по порядоку во входных данных). Если вариантов несколько, вывести какой-либо из них. 
}
\pagebreak
