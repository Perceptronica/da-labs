\section{Описание}
Алгоритмы, предназначенные для решения задач оптимизации, - обычно последовательность шагов, на каждом из которых предоставляется некоторое множество вывборов. Жадный алгоитм - тот, который выбирает самый лучший выбор на данный момент.

Переформулируем задачу так: дана линейная система линейных уравнений (количество уравнений не превышает количество переменных). Требуется найти все уравнения, которые могли бы позволить определить значения переменных и при этом оптимизировать общубю стоимость.

Основная часть алгоритма - применение метода Гаусса (приведение матрицы к ступенчатому виду). На каждом шаге выбираем строку матрицы с минимальной стоимостью; если в какой-то клетке матрицы возникает 0, решения нет - выводим -1.

Если шаг пройден успешно, то выведем индексы строк матрицы в возрастающем порядке.
\pagebreak

\section{Исходный код}
Реализуем алгоритм, описанный в предыдущем пункте.

Опишем основной шаг алгоритма - метод Гаусса:
\begin{enumerate}
	\item ищем минимальную строку по стоимости такую, что элемент по диагонали (по текущему $i$) ненулевой. Если такой строки нет, решения нет;
	\item меняем местами рассматриваемую и найденную строку;
	\item преобразуем к ступенчатому виду строки по текущему pivot-элементу.
\end{enumerate}

\begin{lstlisting}[language=C++]
#include <bits/stdc++.h>

using namespace std;

using vd = vector<double>;
using vi = vector<int>;

int main() {
	ios::sync_with_stdio(false);
	cin.tie(nullptr);
	
	int m, n;
	cin >> m >> n;
	vector<vd> v(m, vd(n + 2, 0));
	for (size_t i = 0; i < m; ++i) {
		for (size_t j = 0; j < n + 1; ++j) {
			cin >> v[i][j];
		}
		v[i][n + 1] = i + 1;
	}
	
	vi min_v;
	for (size_t i = 0; i < n; ++i) {
		int min_cost = INT32_MAX;
		int row = INT32_MIN;
		for (size_t j = i; j < m; ++j) {
			if (v[j][i] != 0 && v[j][n] < min_cost) {
				min_cost = static_cast<int>(v[j][n]);
				row = j;
			}
		}
		if (row == INT32_MIN) { 
			cout << "-1\n"; 
			return 0; 
		}
		swap(v[i], v[row]);
		min_v.push_back(v[i][n + 1]);
		for (size_t j = i + 1; j < m; ++j) {
			double pivot = v[j][i] / v[i][i];
			for (size_t p = 1; p < n; ++p) {
				v[j][p] -= v[i][p] * pivot;
			}
		}
	}
	sort(min_v.begin(), min_v.end());
	for (const auto& elem : min_v) {
		cout << elem << " ";
	}
	cout << "\n";
}	
\end{lstlisting}


\pagebreak

\section{Консоль}
\begin{alltt}
	[anton@home lab8-greedy]$ make
	g++ -O2 -lm -fno-stack-limit -std=c++20 -x c++ solution.cpp -o executable
	[anton@home lab8-greedy]$ ./executable 
	3 3
	1 0 2 3
	1 0 2 4
	2 0 1 2
	-1
	
\end{alltt}
\pagebreak