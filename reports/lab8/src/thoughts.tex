 
\section{Выводы}
Выполнив восьмую лабораторную работу по курсу \enquote{Дискретный анализ}, я познакомился с жадными алгоритмами как способом решения - алгоритмами, заключающихся в принятии локально оптимальных решений на каждом этапе, допуская, что конечное решение также окажется оптимальным.

Применение жадных алгоритмов можно увидеть в задачах на графах, алгоритмах сжатия и других.

Они, как мне показалось, легче понимаются и имплементируются, чем алгоритмы, основанные на динамическое программирование.
\pagebreak