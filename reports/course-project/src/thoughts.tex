 
\section{Выводы}
Я выполнил задание курсового проекта по дискретному анализу.

В ходе выполнения работы был реализован алгоритм быстрого преобразования Фурье (FFT) для анализа аудиосигнала, декодированного из файла формата .mp3. Использовалось скользящее окно размером 4096 отсчётов с шагом 1024, а также оконная функция Ханна для уменьшения спектральных утечек. Бит-реверсивная перестановка и операция "бабочки" обеспечили эффективное вычисление спектра сигнала за время $O(nlogn)$.

Данный подход может быть использован для решения задач анализа звука, таких как распознавание речи, обработка музыки или выделение характерных частотных признаков.

Он отличается, по моему мнению, достаточно сложной для понимания математической базы, но это целиком и полностью оправдано, так как он ускоряет анализ действительных сигналов.
\pagebreak