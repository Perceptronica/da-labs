\section{Описание}
Требуется реализовать алгоритм Укконена построения суффиксного дерева за линейное время.

Алгоритм Укконена в самой простоей реализации имеет сложность $O(n^3)$, так как
добавляет каждый суффикс каждого префикса строки в отличии от добавления всех суффиксов строки. Основная идея в том, чтобы оптимизировать его и получить
сложность $O(n)$.

Заметим, что проще будет продлевать суффиксы на один символ. В некоторых слу-
чаях при продлении можно идти не по всем символам, а сразу по ребру дерева, что
значительно уменьшает число шагов.

Пусть в суффиксном дереве есть строка $x\alpha$ ($x$ - первый символ строки, $\alpha$ - оставшаяся строка), тогда $\alpha$ тоже будет в суффиксном дереве, потому что $\alpha$ является
суффиксом $x\alpha$. Если для строки $x\alpha$ существует некоторая вершина $u$, то существует и вершина $u$ для $\alpha$. Ссылка из $u$ в $v$ называется суффиксной ссылкой.

Суффиксные ссылки позволяют не проходить каждый раз по дереву из корня. Для
построения суффиксных ссылок достаточно хранить номер последней созданной вершины при продлении. Если на этой же фазе мы создаём ещё одну новую вершину, то нужно построить суффиксную ссылку из предыдущей в текущую.

Сложность алгоритма с использованием продления суффиксов и суффиксных ссы-
лок $O(n^2)$.

Для ускорения до $O(n)$ нужно уменьшить объём потребляемой памяти. Будем в каж-
дом ребре дерева хранить не подстроку, а только индекс начала и конца подстроки.

У исходной строки n суффиксов и будет создано не более n внутренних вершин, в
среднем продление суффиксов работает за $O(1)$.

При использовании всех вышеописанных эвристик получим временную и простран-
ственную сложность $O(n)$.

Для нахождения минимального лексикографического разреза строки $s$ построим суффиксное дерево от удвоенной строки и найдём лексикографически минимальный
путь длины $|s|$ в дереве. Сложность $O(n)$.
\pagebreak

\section{Исходный код}
Реализуем суффиксное дерево и алгоритм, описанные в предыдущем пункте. Были произведены некоторые изменения в коде, связанные с реализацией на C++: к примеру, вместо хранения индексов строки хранятся итераторы.

\begin{lstlisting}[language=C++]
#include <bits/stdc++.h>

struct Node;
struct STree;

struct Node {
	std::map<char, Node *> to;
	std::string::iterator begin;
	std::string::iterator end;
	Node *suffixLink;
	
	Node(std::string::iterator begin, std::string::iterator end)
	: begin(begin), end(end), suffixLink(nullptr) {}
	
	~Node() {};
};

struct STree {
	std::string text;
	Node *root;
	Node *activeNode;
	Node *lastAdded;
	std::string::iterator activeEdge;
	int32_t activeLength;
	int32_t remainder;
	
	STree() = default;
	STree(std::string &str)
	: text(str), activeLength(0), activeEdge(text.begin()), remainder(0) {
		root = new Node(text.end(), text.end());
		lastAdded = root;
		activeNode = root;
		root->suffixLink = root;
		
		for (std::string::iterator it = text.begin(); it != text.end(); ++it) {
			addLetter(it);
		}
	}
	
	~STree() { destroy(root); }
	
	void addLetter(std::string::iterator i) {
		lastAdded = root;
		++remainder;
		while (remainder > 0) {
			activeEdge = (activeLength == 0) ? i : activeEdge;
			std::map<char, Node *>::iterator it =
			activeNode->to.find(*activeEdge);
			Node *next;
			if (it == activeNode->to.end()) {
				Node *leaf = new Node(i, text.end());
				activeNode->to[*activeEdge] = leaf;
				lastAdded->suffixLink =
				(lastAdded != root) ? activeNode : lastAdded->suffixLink;
				lastAdded = activeNode;
			} else {
				next = it->second;
				if (checkEdge(i, next)) {
					continue;
				}
				if (*(next->begin + activeLength) == *i) {
					++activeLength;
					lastAdded->suffixLink =
					(lastAdded != root) ? activeNode : lastAdded->suffixLink;
					lastAdded = activeNode;
					break;
				}
				Node *split = new Node(next->begin, next->begin + activeLength);
				Node *leaf = new Node(i, text.end());
				activeNode->to[*activeEdge] = split;
				split->to[*i] = leaf;
				next->begin += activeLength;
				split->to[*next->begin] = next;
				lastAdded->suffixLink =
				(lastAdded != root) ? split : lastAdded->suffixLink;
				lastAdded = split;
			}
			--remainder;
			if (activeNode == root && activeLength > 0) {
				--activeLength;
				activeEdge = i - remainder + 1;
			} else {
				activeNode = activeNode->suffixLink;
			}
		}
	}
	
	void searchLeaves(Node *node, std::vector<int32_t> &answer, int32_t loc) {
		if (node->end == text.end()) {
			answer.push_back(text.size() - loc + 1);
		} else {
			Node *child;
			for (auto it = node->to.begin(); it != node->to.end(); ++it) {
				child = it->second;
				searchLeaves(child, answer, loc + child->end - child->begin);
			}
		}
	}
	
	std::vector<int32_t> search(std::string &str) {
		std::vector<int32_t> answer;
		int32_t loc = 0;
		Node *current = root;
		if (str.length() > text.length()) {
			return answer;
		}
		for (std::string::iterator pos = str.begin(); pos != str.end(); ++pos) {
			auto path = current->to.find(*pos);
			if (path == current->to.end()) {
				return answer;
			}
			current = path->second;
			loc += current->end - current->begin;
			for (std::string::iterator p = current->begin;
			p != current->end && pos != str.end(); ++p, pos++) {
				if (*p != *pos) {
					return answer;
				}
			}
			if (pos == str.end()) {
				break;
			}
			--pos;
		}
		searchLeaves(current, answer, loc);
		sort(answer.begin(), answer.end());
		return answer;
	}
	
	void destroy(Node *node) {
		for (auto it = node->to.begin(); it != node->to.end(); ++it) {
			destroy(it->second);
		}
		delete node;
	}
	
	bool checkEdge(std::string::iterator pos, Node *node) {
		int32_t edgeLength = (pos + 1 < node->end) ? (pos + 1 - node->begin)
		: (node->end - node->begin);
		if (activeLength >= edgeLength) {
			activeEdge += edgeLength;
			activeLength -= edgeLength;
			activeNode = node;
			return true;
		}
		return false;
	}
};

int main(int argc, char *argv[]) {
	std::ios_base::sync_with_stdio(false);
	std::cin.tie(nullptr);
	
	std::string text;
	std::cin >> text;
	text += "$";
	
	STree tree(text);
	int32_t k{0};
	std::vector<int32_t> answer;
	std::string pattern;
	while (std::cin >> pattern) {
		if (pattern.empty()) {
			break;
		}
		++k;
		answer = tree.search(pattern);
		if (!answer.empty()) {
			std::cout << k << ": ";
			for (std::size_t i = 0; i < answer.size(); ++i) {
				std::cout << answer[i] << ((i == answer.size() - 1) ? "\n" : ", ");
			}
		}
	}
	return 0;
}

\end{lstlisting}


\pagebreak

\section{Консоль}
\begin{alltt}
	[anton@home lab5-suffix-tree]$ make
	g++ -O2 -lm -fno-stack-limit -std=c++20 -x c++ solution.cpp -o executable
	[anton@home lab5-suffix-tree]$ ./executable 
	abcdabc
	abcd
	bcd
	bc
	1: 1
	2: 2
	3: 2, 6
	
\end{alltt}
\pagebreak