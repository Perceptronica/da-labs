\section{Описание}
PATRICIA (Practical Algorithm To Retrieve Information In Alphanumeric) trie позволяет эффективно решать следующую задачу - проверять нахождения объектов в дереве только с минимальным количеством проверок (в отличие от compact trie, при поиске объекта в котором необходимо сравнить все биты информации).

PATRICIA имеет следующие свойства:
\begin{enumerate}
	\item Корень - header, у него только левый потомок;
	\item Каждый узел в дереве - слово/объект;
	\item У каждого узла есть номер проверяемого бита (бит header равен 0 или неопределен);
	\item Есть левая и правая ссылки (прямые или обратные). Прямая сслыка гарантирует, что значение проверяемого бита увеличится. Обратные ссылки ведут либо на ту же вершину, либо на элемент сверху в дереве (на пути от header до текущего);
	\item В каждую вершину ведется ровно одна обратная ссылка.
\end{enumerate}

Поиск осуществляется так: идем из хэдера в левый потомок. Если бит ключа равен 0, идем по левой ссылке. Иначе - по правой ссылке. Делаем до момента перехода по обратной ссылке. Проверяем ключ, если он совпадает, мы нашли ключ.

Вставка осуществляется через поиск, вычисление первого несовпавшего бита, создание вершины (и корректное распределение левой и правой ссылки создаваемой вершины) и разбитие ссылки (не наршуая свойство возрастания значения проверяемого бита). Если дерево изначально пустое, вставим хэдер: вершину с одной левой обратной ссылкой на себя.

Удаление имеет три случая:
\begin{enumerate}
	\item Если есть только хэдер или это ребенок, удаляем его.
	\item Если удаляемый элемент имеет только одну на другой элемент (одну прямую ссылку), берем ссылку родителя на него и направляем на ребенка.
	\item Если удаляемый элемент X имеет две ссылки на другие элементы, заменим его ключ на ключ Q - элемента, ссылающегося на X и удаляем Q. Р - элемент, ссылающийся на Q, теперь P ссылается на заменимый элемент.
\end{enumerate}

\pagebreak

\section{Исходный код}
Создадим структуры PATRICIA trie, вершины и методы для них, а также осуществим сериализацию и десериализацию дерева.

\begin{lstlisting}[language=C++]
#include <bits/stdc++.h>

using namespace std;

struct Node;
struct Trie;

using bit_t = int32_t;
using stype = string;
using vtype = uint64_t;
using key = pair<stype, vtype>;
using nptr = Node*;
using tptr = Trie*;

const bit_t MAX_LEN_KEY = 257;
const bit_t BIT_COUNT = 5;

struct Node {
	key k;
	bit_t bit;
	Node* l;
	Node* r;
	vtype num;
	
	Node() {
		k.first = nullptr;
		k.second = -1;
		bit = -1;
		l = this;
		r = this;
	}
	
	Node(stype word, vtype value, bit_t b) : l(this), r(this) {
		k.first = word;
		k.second = value;
		bit = b;
	}
	
	Node(stype word, vtype value, bit_t b, nptr left, nptr right) {
		k.first = word;
		k.second = value;
		bit = b;
		l = left;
		r = right;
	}
	
	~Node() {
		l = nullptr;
		r = nullptr;
		k.first = "\0";
	};
};

bool get_bit(stype& key, bit_t bit_id) {
	bit_t byte_n = bit_id / 8;
	bit_t bit_n = bit_id % 8;
	bit_t symbol;
	if (bit_id < 0 || byte_n >= key.size()) {
		return false;
	} else {
		symbol = key[byte_n];
	}
	return (symbol & (128 >> bit_n)) != 0;
}

bit_t leftmost_bit(stype& key1, stype& key2) {
	bit_t i = 0;
	bit_t max_sz = max(key1.size(), key2.size());
	while (key1[i] == key2[i]) {
		i++;
		if (i == max_sz) {
			return i * 8;
		}
	}
	i = i * 8;
	while (get_bit(key1, i) == get_bit(key2, i)) {
		i++;
	}
	return i;
}

struct Trie {
	nptr header;
	
	Trie() { header = nullptr; }
	
	~Trie() { destructR(header); }
	
	void destructR(nptr n) {
		if (n == nullptr) {
			return;
		} else if (n->l->bit > n->bit) {
			destructR(n->l);
		} else if (n->r != nullptr && n->r->bit > n->bit) {
			destructR(n->r);
		}
	}
	
	nptr search(stype& k) {
		if (header == nullptr) {
			cout << "NoSuchWord\n";
			return nullptr;
		}
		nptr p = this->header;
		nptr c = this->header->l;
		while (p->bit < c->bit) {
			p = c;
			c = get_bit(k, c->bit) ? c->r : c->l;
		}
		if (c->k.first.compare(k) == 0) {
			cout << "OK: " << c->k.second << '\n';
		} else {
			cout << "NoSuchWord\n";
			return nullptr;
		}
		return c;
	}
	
	nptr insert(stype k, vtype d) {
		if (this->header == nullptr) {
			auto newNode = new Node(k, d, 0, nullptr, nullptr);
			newNode->l = newNode;
			this->header = newNode;
			cout << "OK\n";
			return header;
		} else {
			nptr p;
			nptr t;
			nptr x;
			p = this->header;
			t = this->header->l;
			while ((p->bit < t->bit)) {
				p = t;
				t = (get_bit(k, t->bit) ? t->r : t->l);
			}
			if (t->k.first.compare(k) == 0) {
				cout << "Exist\n";
				return nullptr;
			}
			bit_t i = leftmost_bit(k, t->k.first);
			// cout << "!: " << i << "\n";
			p = this->header;
			x = this->header->l;
			auto newNode = new Node(k, d, i, nullptr, nullptr);
			while ((p->bit < x->bit) && (x->bit < newNode->bit)) {
				p = x;
				x = (get_bit(k, x->bit) ? x->r : x->l);
			}
			newNode->l = (get_bit(k, i) ? x : newNode);
			newNode->r = (get_bit(k, i) ? newNode : x);
			if (x == p->r) {
				p->r = newNode;
			} else {
				p->l = newNode;
			}
			cout << "OK\n";
			return t;
		}
	}
	
	bool remove(stype& k) {
		if (header == nullptr) {
			cout << "NoSuchWord\n";
			return false;
		}
		if (header->l == header) {  // header is an only node
			if (header->k.first.compare(k) == 0) {
				delete header;
				header = nullptr;
				cout << "OK\n";
				return true;
			} else {
				cout << "NoSuchWord\n";
				return false;
			}
		}
		nptr prePrev = nullptr, parOfPrev = nullptr, previous = header,
		current = header->l;
		while (previous->bit < current->bit) {
			prePrev = previous;
			previous = current;
			current = get_bit(k, current->bit) ? current->r : current->l;
		}
		if ((current->k.first).compare(k) != 0) {
			std::cout << "NoSuchWord\n";
			return false;
		}
		if (current == previous) {
			if (prePrev->l == current) {
				prePrev->l = (current->l == current ? current->r : current->l);
			} else {
				prePrev->r = (current->l == current ? current->r : current->l);
			}
			delete current;
			std::cout << "OK\n";
			return true;
		}
		nptr p = current, q = previous, qPar = prePrev, r = nullptr;
		current = header->l;
		r = header;
		while (r->bit < current->bit) {
			r = current;
			current = get_bit(q->k.first, current->bit) ? current->r : current->l;
		}
		bool flag = get_bit(r->k.first, q->bit);
		if (r->r == q) {
			r->r = p;
		} else {
			r->l = p;
		}
		if (qPar->r == q) {
			qPar->r = flag ? q->r : q->l;
		} else {
			qPar->l = flag ? q->r : q->l;
		}
		p->k.first = q->k.first;
		p->k.second = q->k.second;
		delete q;
		std::cout << "OK\n";
		return true;
	}
	
	int save(stype& filename) {
	  // look solution
	}
	int saveR(nptr node, std::ofstream& file) {
	  // look solution
	}
	int load(string filename) {
	  // look solution
	}
};

void to_lowercase(stype& s) {
	for (size_t i = 0; s[i] != '\0'; ++i) {
		if (s[i] >= 'A' && s[i] <= 'Z') {
			s[i] = s[i] - 'A' + 'a';
		}
	}
}

	
\end{lstlisting}
\pagebreak

\section{Консоль}
\begin{alltt}
[anton@home lab2-patricia]$ make
g++ -O2 -lm -fno-stack-limit -std=c++20 -x c++ solution.cpp -o executable
[anton@home lab2-patricia]$ cat test/test2.in
a
+ a 1
A
a
- \ a
A
+ aa 2
aa
+ b 3
b
- \ aa
b
[anton@home lab2-patricia]$ make run < test/test2.in
./executable
NoSuchWord
OK
OK: 1
OK: 1
OK
NoSuchWord
OK
OK: 2
OK
OK: 3
OK
OK: 3

\end{alltt}
\pagebreak

