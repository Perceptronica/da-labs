 
\section{Выводы}
Выполнив первую лабораторную работу по курсу \enquote{Дискретный анализ}, я познакомился с поразрядной сортировкой и сортировкой подсчета, которая является устойчивой и работает за линейное время, но требует $O(k)$ дополнительной памяти, что может сказаться на производительности при большом объеме входных данных. 

Также существует небольшие проблемы использования такой сортировки на небольших по объему входных данных: во-первых, встроенные в STL функции сортировки работают в таком случае эффективнее и по памяти, и по времени; во-вторых, если в данных всего лишь несколько элементов с большой разницей между минимумов по значению и максимумом, то создается массив, состоящий из большого количества нулей; в-третьих, неудобно сортировать элементы, которые не являются/нельзя преобразовать в целочисленные элементы.

Также я вспомнил про применение отключения синхронизации стандартных потоков C и C++, которое я использовал в контексте олимпиадного программирования.

Поразрядная сортировка - довольно мощный и хороший метод сортировки большого количества объектов, несмотря на большое использование памяти.
\pagebreak