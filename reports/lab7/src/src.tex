\section{Описание}
Исходную строку обозначим $S$, количество способов вычеркиваний подстроки $S[i, j]$ - $d(S[i, j])$.

Начнем решать задачу с простых подпоследовательностей. Однобуквенная подпоследовательность является палиндромом, поэтому $dp(S[i, i]) = 1$.

Для двухбуквенной последовательности $S[i, i + 1]$ может существовать два варианта ответа: 
\begin{enumerate}
	\item $d(S[i, i+1]) = 2$, когда $S[i] \neq S[i + 1]$;
	\item $d(S[i, i+1]) = 3$, когда $S[i] = S[i + 1]$.
\end{enumerate}

Это база ДП. Мы можем распространить этот принцип на подстроки длины больше 2:
\begin{enumerate}
	\item $d(S[i, j]) = d(S[i + 1, j]) + d(S[i, j -1]) + 1$, если $S[i] = S[j]$;
	\item $d(S[i, j]) = d(S[i + 1, j]) + d(S[i, j -1]) - d(S[i + 1, j - 1])$, если $S[i] \neq S[j]$.
\end{enumerate}

Заполним матрицу $D$ этими значениями. Получится матрица с нулями ниже главной диагонали и ненулевыми значениями сверху (поэтому мы можем распространить принцип на более сложные случаи). Ответ находится в клетке $D[0][|S|-1]$.
\pagebreak

\section{Исходный код}
Реализуем алгоритм, описанный в предыдущем пункте, считав строку, обработав простой случай  и распространив вычисления на другие (более сложные) случаи.

\begin{lstlisting}[language=C++]
#include <bits/stdc++.h>

using namespace std;

int main() {
	string s;
	cin >> s;
	size_t s_sz = s.size();
	uint64_t D[s_sz][s_sz];
	for (size_t i = 0; i < s_sz; ++i) {
		for (size_t j = 0; j < s_sz; ++j) {
			if (i == j) { // one char string is a palindrome
				D[i][j] = 1;
			} else if (i > j) {
				D[i][j] = 0;
			}
		}
	}
	
	for (size_t len = 2; len <= s_sz; ++len) {
		for (size_t i = 0; i <= s_sz - len; ++i) {
			size_t j = i + len - 1;
			if (s[i] == s[j]) {
				D[i][j] = D[i + 1][j] + D[i][j - 1] + 1;
			} else {
				D[i][j] = D[i + 1][j] + D[i][j - 1] - D[i + 1][j - 1];
			}
		}
	} 
	
	cout << D[0][s_sz - 1] << endl;
}

	
	
\end{lstlisting}


\pagebreak

\section{Консоль}
\begin{alltt}
	[anton@home lab7-dp]$ make
	g++ -O2 -lm -fno-stack-limit -std=c++20 -x c++ solution.cpp -o executable
	[anton@home lab7-dp]$ ./executable 
	BAOBAB
	22
	
\end{alltt}
\pagebreak