 
\section{Выводы}
Выполнив седьмую лабораторную работу по курсу \enquote{Дискретный анализ}, я познакомился с динамическим программированием как способом решения различных задач. Он позволяет ускорять решения при помощи разбиения задачи на более мелкие независимые. При этом необходимо создать оптимальные подзадачи.

Процесс разработки алгоритмов динамического программирования состоит из следующих шагов:
\begin{enumerate}
	\item описание структуры оптимального решения;
	\item рекурсивное определение значения оптимального решения;
	\item вычисление значения с помощью метода восходящего анализа;
	\item составление решения на основе оптимального решения.
\end{enumerate}

Разработанный алгоритм был идейно похож на алгоритм решения задачи о наибольшей подпоследовательности-палиндроме, поэтому было интересно применить идею к решению лабораторной работы.
\pagebreak