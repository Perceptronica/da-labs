\section{Описание}
Компонентой связности неориентированного графа называется подмножество вершин, достижимых из какой-то заданной вершины. Как следствие неориентированности, все вершины компоненты связности достижимы друг из друга.

Дан неориентированный граф $G$ с $n$ вершинами и $m$ рёбрами. Требуется найти в нём все компоненты связности, то есть разбить вершины графа на несколько групп так, что внутри одной группы можно дойти от одной вершины до любой другой, а между разными группами путей не существует.

Для решения поставленной задачи модифицируем обход в глубину.

Поиск (обход) в глубину (DFS, depth-first search) - рекурсивный алгоритм обхода корневого дерева или графа, начинающийся в корневой вершине (в случае графа - произволной вершине) и рекурсивно обходящий весь граф, посещая вершину ровно один раз.

Модификация заключается в дополнении вектора, отвечающего за хранение всех вершин какой-то из компонент, и добавлении аргумента номера компоненты для добавления вершины.

Для корректного вывода ответа воспользуемся стандартным методом сортировки sort(). Все компоненты связности выводятся в правильном порядке, так как запускается цикл для вызова DFS и берущий вершины по возрастающему порядку.

\begin{lstlisting}[language=C++]
#include <bits/stdc++.h>

using namespace std;
const uint32_t MAXN = 1e5 + 1;
array<bool, MAXN> visited;
vector<vector<uint32_t>> answer;
vector<vector<uint32_t>> g(MAXN, vector<uint32_t>(0)); 

void dfs(uint32_t v, int32_t c) {
	visited[v] = true;
	for (uint32_t u : g[v]) {
		if(!visited[u]) {
			dfs(u, c);
		}
	}
	answer[c].push_back(v);
}

int main() {
	ios_base::sync_with_stdio(false);
	cin.tie(nullptr);
	uint32_t n, m;
	cin >> n >> m;
	uint32_t a, b;
	for (uint32_t i = 0; i < m; ++i) {
		cin >> a >> b;
		g[a].push_back(b);
		g[b].push_back(a);
	}
	int32_t num = -1;
	for (uint32_t v = 1; v <= n; ++v) {
		if (!visited[v]) {
			answer.push_back({});
			dfs(v, ++num);
		}
	}
	for (auto& v : answer) {
		sort(v.begin(), v.end());
		for (const auto& u: v) {
			cout << u << ' ';
		}
		cout << '\n';
	}
}
\end{lstlisting}


\pagebreak

\section{Консоль}
\begin{alltt}
	[anton@home lab9-graphs]$ make
	g++ -O2 -lm -fno-stack-limit -std=c++20 -x c++ solution.cpp -o executable
	[anton@home lab9-graphs]$ ./executable
	1 1
	1 1
	1 
	[anton@home lab9-graphs]$ ./executable
	2 2
	1 1
	2 2
	1 
	2 
	[anton@home lab9-graphs]$ ./executable
	6 5
	2 5
	4 6
	6 3
	3 1
	4 3
	1 3 4 6 
	2 5
	[anton@home lab9-graphs]$ ./executable
	5 4
	1 2
	2 3
	1 3
	4 5
	1 2 3
	4 5
	
\end{alltt}
\pagebreak